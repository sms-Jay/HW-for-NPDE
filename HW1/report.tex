\documentclass[11pt,a4 paper,one side]{article}
\usepackage{amsmath,amssymb,graphicx}
\usepackage{ctex}  
\usepackage[colorlinks=true,linkcolor=red,citecolor=red,filecolor=magenta,urlcolor=cyan]{hyperref}
\usepackage{bookmark}
\usepackage{fontspec}
\setmainfont{Times New Roman}
\usepackage{xcolor}
\usepackage{geometry}
\geometry{a4paper, left=2.5cm, right=2.5cm, top=2.5cm, bottom=2.5cm}
\title{偏微分方程数值解+第一次上机作业}
\author{2100012131 蒋鹏}
\date{\today}
\begin{document}
\maketitle
\tableofcontents
\section{问题描述}
实现稀疏矩阵的CSR存储格式(Compressed Sparse Row),需要实现的核心功能包括:
\par 1.该存储格式的主要元素,包括:非零元个数,矩阵行列,非零元行列索引以及元素值
\par 2.构造稀疏矩阵(可以以2维Dirichlet边界Poisson方程系数矩阵为例)
\par 3.基本操作:矩阵向量乘法,取矩阵上,下三角部分,取对角线部分(服务于迭代Jacobi等线性方程组求解方法)
\par 另外我们自己设计了一些其他功能:输出稀疏矩阵的matrix form,修改某指定位置的元素(不改变非零元位置)。
\section{算法设计}
\subsection{CSR存储格式}
首先存储稀疏矩阵的基本信息,包括矩阵的行数$row$,列数$col$以及非零元的个数$num$。然后,
我们采取三元组法来存储稀疏矩阵,即用三个一维数组来分别存储非零元的值、列索引以及行指针。具体来说,我们定义以下三个数组:
\par 1.值数组$values$:存储矩阵中所有非零元的值,长度为$num$。
\par 2.列索引数组$col\_idx$:存储每个非零元所在的列索引,长度为$num$。
\par 3.行指针数组$row\_ptr$:存储当前行的第一个非零元在值数组中的位置,长度为$row+1$。其中,$row\_ptr[i]$表示第$i$行的第一个非零元在$values$数组中的位置,
于是$row\_ptr[i]-row\_ptr[i-1]$即为第$i$行的非零元个数。($1\leq i \leq row$)
\subsection{Laplacian矩阵构造}
对于二维Dirichlet边界Poisson方程,我们使用五点差分格式来离散化该方程,从而得到一个稀疏矩阵。假设我们在一个$(n+2) \times (n+2)$的网格上进行离散化,
需要求解的内部节点为$n \times n$个,因此矩阵的大小为$n^2 \times n^2$。每个内部节点与其上下左右四个邻居节点相连,进行naive的扫描构造矩阵,注意特殊点的处理即可。
\subsection{矩阵向量乘法}
矩阵向量乘法的实现比较简单,我们只需要遍历每一行的非零元,然后将对应的值与向量中的元素相乘并累加即可。具体来说,对于第$i$行($0\leq i \leq row-1$),我们从$row\_ptr[i]$到$row\_ptr[i+1]-1$遍历所有非零元,
然后将这些非零元与向量中的对应元素相乘并累加,最终得到结果向量的第$i$个元素。
\subsection{取矩阵的上、下三角、对角部分}
取矩阵的上、下三角以及对角部分的实现也比较简单。以上三角矩阵为例,我们只需要遍历每一行的非零元,然后将列索引大于行索引的非零元保留下来即可。于是我们完成了分解:A=L+U+D,这将服务于设计迭代法。
\section{数值实验}
首先测试CSR格式的读取、存储、输出,顺便验证矩阵向量乘法功能以及取上三角、下三角、对角部分。
测试输入为文件SpM\_input.txt,向量$x$为全1向量。
\par 其次测试Laplacian矩阵的构造功能,测试数据为n=5。
\par 数值结果见output.txt。
\end{document}