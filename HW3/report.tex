\documentclass[11pt,a4 paper,one side]{article}
\usepackage{amsmath,amssymb,graphicx,subcaption}
\usepackage{ctex}  
\usepackage[colorlinks=true,linkcolor=red,citecolor=red,filecolor=magenta,urlcolor=cyan]{hyperref}
\usepackage{bookmark}
\usepackage{fontspec}
\setmainfont{Times New Roman}
\usepackage{xcolor}
\usepackage{geometry}
\geometry{a4paper, left=2.5cm, right=2.5cm, top=2.5cm, bottom=2.5cm}
\title{偏微分方程数值解+第三次上机作业}
\author{2100012131 蒋鹏}
\date{\today}
\begin{document}
\maketitle
\tableofcontents
\section{问题描述}
在指定区域上$\Omega$求解Dirichlet边值问题的热方程
\begin{align}
\begin{cases}
    \frac{\partial u}{\partial t}=\Delta u+\rho(t)\delta(x_0,y_0), &(x,y)\in \Omega,t>0 \\
    u(x,y,0)=u_0(x,y),&(x,y)\in \bar{\Omega}\\
    u(x,y,t)=g(x,y,t), &(x,y)\in \partial \Omega,t>0
\end{cases}
\end{align}
其中,$u_0(x,y),g(x,y,t)$是给定的函数且满足相容性条件,$\delta(x)$是Dirac函数,$(x_0,y_0)$选取为$\Omega$内一点,
$\rho(t)$举例可取为$\rho(t)=\sin(t)$。区域$\Omega$如图\ref{Domain}所示。\begin{figure}
    \centering
    \includegraphics[width=0.5\linewidth]{Domain.png}
    \caption{Domain}
    \label{Domain}
\end{figure}
\section{网格离散}
仍然采取最朴素的均匀网格离散,与HW2一致。如图\ref{Grid}所示。对于数值解,$U_{i,j}^{n}\approx u(x_i,y_i,t_n)$。\begin{figure}
    \centering
    \includegraphics[width=0.5\linewidth]{Grid.jpg}
    \caption{Grid}
    \label{Grid}
\end{figure}
\section{数值格式}

\subsection{全局热方程初值问题的精确解}
对于$n$维热方程初值问题\begin{equation}
    \begin{cases}
        u_t=\Delta u + f(x,t),x\in \mathbb{R}^n,t>0\\
        u(x,0)=u_0(x),x\in \mathbb{R}^n
    \end{cases}
\end{equation}
定义$n$维热核\begin{equation}
    K(x,t)=\frac{1}{(4\pi t)^{n/2}}e^{-\frac{|x|^2}{4t}},x\in \mathbb{R}^n,t>0
\end{equation}
则该初值问题具有解\begin{equation}
    u(x,t)=\int_{\mathbb{R}^n} K(x-\xi,t)u_0(\xi)d\xi + \int_{0}^{t}\int_{\mathbb{R}^n}K(x-\xi,t-\tau)f(\xi,\tau)d\xi d\tau
\end{equation}
在此题目中,利用$\delta$函数的性质,可以得到初值问题的解为\begin{equation}
    u(x,y,t)=\int_{\mathbb{R}^2}K(x-\xi,y-\eta,t)u_0(\xi,\eta) d\xi d\eta +
             \int_{0}^{t}K(x-x_0,y-y_0,t-\tau)\rho(\tau)d\tau
\end{equation}
\subsection{有限差分法}
使用有限差分法存在无法离散化$\delta(x_0,y_0)$的问题,因为这是一个广义函数,而在实际物理世界中并不存在严格意义上的$\delta$函数。
根据物理定义,$\delta$函数可以视为一列高斯分布或者一列方波脉冲的极限。根据此极限,我们对方程采取正则化策略:定义
\begin{equation}
    \delta_{\sigma}(x,y)=\frac{1}{2\pi \sigma^2}e^{\frac{|x-x_0|^2+|y-y_0|^2}{2\sigma^2}}\approx \delta(x_0,y_0)
\end{equation}
然后离散对应热方程
\begin{align}
\begin{cases}
    \frac{\partial u_{\sigma}}{\partial t}=\Delta u_{\sigma}+\rho(t)\delta_{\sigma}(x,y), &(x,y)\in \Omega,t>0 \\
    u_{\sigma}(x,y,0)=u_0(x,y),&(x,y)\in \bar{\Omega}\\
    u_{\sigma}(x,y,t)=g(x,y,t), &(x,y)\in \partial \Omega,t>0
\end{cases}
\end{align}

对此正则化之后的方程,用上述网格,进行有限差分离散,其中选取参数$\sigma=\mathcal{O}(h)$来模拟一系列正态分布收敛到极限$\delta$函数的过程。
定义第$n$步的矩阵向量乘法操作为$A^n$。令$\mu = \frac{dt}{h^2}$。
\par 对于内点,进行五点中心差分离散,即\begin{equation}
    A^n U_{i,j}^n=\frac{\Delta_h U_{i,j}^n}{h^2}=\frac{U_{i+1,j}^n+U_{i-1,j}^n+U_{i,j+1}^n+U_{i,j-1}^n-4U_{i,j}^n}{h^2}
\end{equation}
对于在边界上的点,赋予Dirichlet边界条件,即\begin{equation}
    A^n U_{i,j}^n = g(x_i,y_j,t_n)
\end{equation}
对于临界边界上的点,即在右边界$y=2x-1(1\leq x \leq 2)$左侧的一些点,用五点偏心差分离散\begin{equation}
    A^n U_{i,j} = \frac{\Delta_h U_{i,j}^n}{h^2}=\frac{U_{i+1,j}^n+U_{i-1,j}^n-2U_{i,j}^n
    +8/3g(x_{i+1/2},y_j)+4/3U_{i,j-1}^n-4U_{i,j}^n}{h^2}
\end{equation}
\par 显式格式:时间方向上用一阶前向Euler,对方程做离散可以得到\begin{equation}
    \frac{U_{i,j}^{n+1}-U_{i,j}^n}{dt}=\frac{\Delta_h U_{i,j}^n}{h^2}+\rho(t_n)\delta_{\sigma}(x_i,y_j)
\end{equation}
更新格式\begin{equation}
U_{i,j}^{n+1}=
\begin{cases}
dt(\frac{\Delta_h U_{i,j}^n}{h^2}+\rho(t_n)\delta_{\sigma}(x_i,y_j))+U_{i,j}^n,\text{内点}\\
g(x_i,y_j,t_{n+1}),\text{边界点}\\
\frac{U_{i,j-1}^{n+1}+2g(x_{i+1/2},y_j)}{3}
\end{cases}
\end{equation}
\par 隐式格式:时间方向上用一阶后向Euler,对方程做离散可以得到\begin{equation}
    \frac{U_{i,j}^{n+1}-U_{i,j}^n}{dt}=\frac{\Delta_h U_{i,j}^{n+1}}{h^2}+\rho(t_{n+1})\delta_{\sigma}(x_i,y_j)
\end{equation}
即\begin{equation}
    (I-\mu \Delta_h)U_{i,j}^{n+1}=U_{i,j}^{n}+dt*\rho(t_{n+1})\delta_{\sigma}(x_i,y_j)
\end{equation}
定义不同点的矩阵向量乘法\begin{equation}
    AU_{i,j}=
    \begin{cases}
        (I-\mu \Delta_h)U_{i,j},\text{内点和临边界点}\\
        U_{i,j},\text{边界点}\\
    \end{cases}
\end{equation}
对于临边界点,要将$\Delta_h$所含常数均移动到方程右侧,最后形成向量\begin{equation}
    b_{i,j}=\begin{cases}
        U_{i,j}^{n}+dt\rho(t_{n+1})\delta_{\sigma}(x_i,y_j),\text{内点}\\
        U_{i,j}^{n}+dt\rho(t_{n+1})\delta_{\sigma}(x_i,y_j)+8/3\mu g(x_{i+1/2},y_j),\text{临边界点}\\
        g(x_i,y_j),\text{边界点}
    \end{cases}
\end{equation}
这样最终得到了线性方程组$AU^{n+1}=b$和相应的矩阵向量乘法操作,利用GMRES进行求解。
\subsection{有限体积法}
注意,本方法适用于在方形区域、均匀网格上进行求解,在本题所给区域的测试效果并不好,因为区域不规则,控制体选取与通量选取很难保证某种守恒性。但我们还是展示该方法,测试区域为$[0,1]*[0,1]$。在每个内点,取所在单元格为控制体;利用$\delta(x_0,y_0)$在全局积分为1,得到:\begin{align}
    u_{i,j}'(t)&\approx \frac{d}{dt}\frac{\int_{x_{i-1/2}}^{x_{i+1/2}}\int_{y_{j-1/2}}^{y_{j+1/2}}u(x,y,t)dxdy}{h^2}\\
    &=\frac{\int_{x_{i-1/2}}^{x_{i+1/2}}\int_{y_{j-1/2}}^{y_{j+1/2}}u_t(x,y,t)dxdy}{h^2}\\
    &=\frac{\int_{x_{i-1/2}}^{x_{i+1/2}}\int_{y_{j-1/2}}^{y_{j+1/2}}  (\Delta u+\rho(t)\delta (x_0,y_0))  dxdy}{h^2}\\
    &=\frac{1}{h^2}(\int_{x_{i-1/2}}^{x_{i+1/2}} (u_y(x,y_{j+1/2},t) - u_y(x,y_{j-1/2},t)) dx)\\
    &+\frac{1}{h^2}(\int_{y_{j-1/2}}^{y_{j+1/2}} (u_x(x_{i+1/2},y,t) - u_x(x_{i-1/2},y,t)) dy)\\
    &+\frac{1}{h^2}(\rho(t)\Delta (x_0,y_0))\\
    &\approx \frac{1}{h^2}((u_y(x_i,y_{j+1/2},t) - u_y(x_i,y_{j-1/2},t)))\\
    &+\frac{1}{h^2}h(u_x(x_{i+1/2},y_j,t) - u_x(x_{i-1/2},y_j,t)) \\
    &+\frac{1}{h^2}\rho(t)\Delta (x_0,y_0)\\
\end{align}
其中\begin{equation}
    \Delta(x_0,y_0)=\begin{cases} 1,(x_0,y_0)\in \text{(i,j)的控制体} \\ 0,else
    \end{cases}
\end{equation}
单元控制体每条边上的法向导数的近似:
\par 若该条边相邻单元仍是一个控制体,即该边不是边界:以右侧为例,则进行中心差分近似,即
\begin{equation}
    u_x(x_{i+1/2},y_j,t)\approx \frac{U_{i+1,j}-U_{i,j}}{h}
\end{equation}
\par 若该条边是边界:以底边界为例,则进行偏心差分近似,即
\begin{equation}
    u_x(x_{i},y_{j-1/2},t)\approx \frac{2U_{i,j}-2g(x_i,y_{j-1/2})}{h}
\end{equation}
在进行这样的数值离散后,有限体积格式可以写作
\begin{equation}
    U_{i,j}'=\frac{\Delta_h U_{i,j}+\rho(t)\Delta(x_0,y_0)}{h^2}
\end{equation}
其中$\Delta_h$在不同点有不同表达式,这由上述对不同类型边上的法向导数的近似可以得到。
\par 显式格式:时间方向上用一阶前向Euler,对方程做离散可以得到\begin{equation}
    \frac{U_{i,j}^{n+1}-U_{i,j}^n}{dt}=\frac{\Delta_h U_{i,j}^n+\rho(t_n)\Delta(x_0,y_0)}{h^2}
\end{equation}
更新格式\begin{equation}
U_{i,j}^{n+1}=\mu(\Delta_h U_{i,j}^n+\rho(t_n)\Delta(x_0,y_0))+U_{i,j}^n
\end{equation}
\par 隐式格式:时间方向上用一阶后向Euler,对方程做离散可以得到\begin{equation}
    \frac{U_{i,j}^{n+1}-U_{i,j}^n}{dt}=\frac{\Delta_h U_{i,j}^{n+1}+\rho(t_{n+1})\Delta(x_0,y_0)}{h^2}
\end{equation}
即\begin{equation}
    (I-\mu \Delta_h)U_{i,j}^{n+1}=U_{i,j}^{n}+\mu\rho(t_{n+1})\Delta(x_0,y_0)
\end{equation}
注意,对于本身是边界的边上,对应的方程需要把含有$g$的量移到方程右边,最终得到向量
\begin{equation}
    b_{i,j}=\begin{cases}
        U_{i,j}^{n}+\mu\rho(t_{n+1})\Delta(x_0,y_0),\text{内点}\\
        U_{i,j}^{n}+\mu\rho(t_{n+1})\Delta(x_0,y_0)+2\mu g(x_i,y_{j-1/2}),\text{以下边界、非角点为例}\\
        U_{i,j}^{n}+\mu\rho(t_{n+1})\Delta(x_0,y_0)+2\mu (g(x_i,y_{j-1/2})+g(x_{i-1/2},y_j)),\text{以左下角角点为例}\\
    \end{cases}
\end{equation}
这样最终得到了线性方程组$AU^{n+1}=b$和相应的矩阵向量乘法操作,利用GMRES进行求解。
\section{数值算例}
若指定初值条件$u_0(x,y)$,则由解表达式可以给出任意时刻的Dirichlet边界条件$g(x,y,t)$和真实解,这便需要进行数值积分,因此我们编写了Simpson数值积分与热方程真实解的表达式作为头文件。
为了减小运算量,取测试算例,设定初值条件恒为0,$\rho(t)=\sin (t)$,求解时间为$t_{end}=0.1$,点热源坐标为$(x_0,y_0)=(0.985,0.211)$。FDM算法中设置$\sigma = 0.9h$.
\section{数值结果展示分析}
以表格\ref{数值结果展示}展示所有的数值结果,以FDM的结果为主。
\begin{table}
    \centering
    \begin{tabular}{c c c c}
        \hline
        求解器&$dt$&$h$&$\|e_h\|_{2}$\\
        \hline
        FDM-forward-Euler & $0.25h^2$ & 0.1& 2.55e-3\\
        &&0.05&8.15e-4\\
        &&0.025&3.15e-4\\
        &&0.0125&2.80e-4\\
        &&0.00625&8.90e-5\\
        &&0.003125&4.42e-5\\
        \hline
        FDM-backward-Euler & $h$ & 0.1 & 2.99e-3\\
        &&0.05&1.41e-3\\
        &&0.025&7.17e-4\\
        &&0.0125&4.37e-4\\
        &&0.00625&1.95e-4\\
        &&0.003125&9.82e-5\\
        \hline
        FDM-backward-Euler & $h^2$ & 0.1 & 2.80e-3\\
        &&0.05&8.76e-4\\
        &&0.025&3.27e-4\\
        &&0.0125&2.81e-4\\
        &&0.00625&9.00e-5\\
        \hline
        FVM-forward-Euler & $0.2h^2$ & 0.1 & 2.09e-3\\
        && 0.05 & 7.71e-4\\
        && 0.025 & 2.54e-4\\
        && 0.0125 & 3.48e-4\\
        &&0.00625&1.02e-4\\
        &&0.003125&6.04e-5\\
        \hline
        FVM-backward-Euler & $h$ & 0.1 & 3.24e-3\\
        && 0.05&1.51e-3\\
        && 0.025&6.97e-4\\
        && 0.0125&4.91e-4\\
        && 0.00625&2.02e-4\\
        &&0.003125&1.06e-4\\
        \hline
        FVM-backward-Euler & $h^2$ & 0.1 & 1.96e-3\\
        && 0.05&7.62e-4\\
        && 0.025&2.68e-4\\
        && 0.0125&3.47e-4\\
        && 0.00625&1.02e-4\\
        \hline
    \end{tabular}
    \caption{数值结果展示}
    \label{数值结果展示}
\end{table}
为了展示收敛性,对这些数据作$\log(h)-\log(e_h)$线性回归,结果如图\ref{数值结果}所示。
\begin{figure}[htbp]
    \centering
    \begin{subfigure}{0.45\textwidth}
         \includegraphics[width=\textwidth]{FDM-forward-dt=0.25h^2.png}
        \caption{$FDM-forward-dt=0.25h^2$}
        \label{FDM-forward-dt=0.25h^2}
    \end{subfigure}
    \begin{subfigure}{0.45\textwidth}
         \includegraphics[width=\textwidth]{FDM-backward-dt=h.png}
        \caption{$FDM-backward-dt=h$}
        \label{FDM-backward-dt=h}
    \end{subfigure}
    \vspace{0.5cm}
    \begin{subfigure}{0.45\textwidth}
         \includegraphics[width=\textwidth]{FVM-forward-dt=0.20h^2.png}
        \caption{$FVM-forward-dt=0.20h^2$}
        \label{FVM-forward-dt=0.20h^2}
    \end{subfigure}
    \begin{subfigure}{0.45\textwidth}
         \includegraphics[width=\textwidth]{FVM-backward-dt=h.png}
        \caption{$FVM-backward-dt=h$}
        \label{FVM-backward-dt=h}
    \end{subfigure}
    \vspace{0.5cm}
    \begin{subfigure}{0.45\textwidth}
         \includegraphics[width=\textwidth]{FDM-backward-dt=h^2.png}
        \caption{$FDM-backward-dt=h^2$}
        \label{FDM-backward-dt=h^2}
    \end{subfigure}
    \begin{subfigure}{0.45\textwidth}
         \includegraphics[width=\textwidth]{FVM-backward-dt=h^2.png}
        \caption{$FVM-backward-dt=h^2$}
        \label{FVM-backward-dt=h^2}
    \end{subfigure}
    \caption{数值结果}
    \label{数值结果}
\end{figure}
从图\ref{数值结果}中可以看出\begin{itemize}
    \item 所有测试算例在一定区间内都表现出接近1阶的收敛性,有的略高于1阶,有的略低于1阶,我们可以近似认为\begin{equation}
    \|e_h\|_2=\mathbb{O}(h)
    \end{equation}
    \item 无论是FDM还是FVM,选取$dt=\mathbb{O}(h)$或者$dt=\mathbb{O}(h^2)$,都只表现出1.0阶的收敛性,这说明两种算法的收敛阶都被其他因素所限制。
    \item 对于FDM,误差由三部分组成:$\delta$函数正则化带来的正则化误差,数值求解正则化方程带来的数值误差,以及用数值积分表示原方程真实解的数值误差。
    由于方程的真实解在$(x_0,y_0)$附近具有对数奇异性,正则化之后的解函数全局有界,无法准确模拟真实解在点热源附近的奇异行为,造成的正则化误差较大。
    同时由于区域的不规则性,偏心差分的设计也会引入边界误差,导致影响全局误差。最后对于后向Euler,当网格加密,求解耗时和矩阵条件数迅速增大,影响解的精度。
    \item 对于FVM,误差由三部分组成:有限体积格式的系统误差,数值求解方程带来的数值误差,以及用数值积分表示原方程真实解的数值误差。
    首先可以将有限体积格式的推导视为将$\delta$函数近似为所在控制体上的均匀分布,也就是紧支集上的“脉冲”,也可以理解为将方程正则化。
    由于方程的真实解在$(x_0,y_0)$附近具有对数奇异性,之后的解函数全局有界,无法准确模拟真实解在点热源附近的奇异行为,造成的正则化误差较大。
    同时由于区域的不规则性,对于非方形的控制体,其单元中心、边界通量等计算都存在偏差,难以同时保持代码的简洁与格式的某种通量守恒性,
    在靠近边界的偏心差分的设计也会引入边界误差,导致影响全局误差。最后对于后向Euler,当网格加密,求解耗时和矩阵条件数迅速增大,影响解的精度。
    \item 观察取$dt=\mathbb{O}(h^2)$,当$h$从0.025下降到0.0125时的现象:在FVM方法中,此时误差发生了反向增长;而在FDM中,虽说误差没有增长,但下降速度在此处最慢。
    而取$dt=O(h)$此现象不明显。分析其原因是:当$h$从0.025下降到0.0125,方程的正则化误差并未发生明显下降,误差项的主导系数几乎不变。
\end{itemize}
\section{总结与反思}
在一定的离散范围和精度要求之内,两种方法均得到了比较好的实现,从理论分析上看,有限体积法应当表现更好,更符合物理实际,
但从实际数值效果上课有限差分法略优于有限差分法,因为复杂的区域边界不利于有限体积的格式设计。下一步,我将再分别从理论分析与实现的层面,再次检查有限体积法的实现是否存在被忽视了的问题。如得到改善,将再次提交。
具体想法包括:在规则方形区域上对两种方法进行测试,再推广到凸多边形区域上进行测试;查阅相关文献对于点热源的正确处理方式以及对于不规则区域的有限体积格式。
\end{document}