\documentclass[11pt,a4 paper,one side]{article}
\usepackage{amsmath,amssymb,graphicx}
\usepackage{ctex}  
\usepackage[colorlinks=true,linkcolor=red,citecolor=red,filecolor=magenta,urlcolor=cyan]{hyperref}
\usepackage{bookmark}
\usepackage{fontspec}
\setmainfont{Times New Roman}
\usepackage{xcolor}
\usepackage{geometry}
\geometry{a4paper, left=2.5cm, right=2.5cm, top=2.5cm, bottom=2.5cm}
\title{偏微分方程数值解+第三次上机作业}
\author{2100012131 蒋鹏}
\date{\today}
\begin{document}
\maketitle
\tableofcontents
\section{问题描述}
在指定区域上$\Omega$求解Dirichlet边值问题的热方程
\begin{align}
\begin{cases}
    \frac{\partial u}{\partial t}=\Delta u+\rho(t)\delta(x_0,y_0), &(x,y)\in \Omega,t>0 \\
    u(x,y,0)=u_0(x,y),&(x,y)\in \bar{\Omega}\\
    u(x,y,t)=g(x,y,t), &(x,y)\in \partial \Omega,t>0
\end{cases}
\end{align}
其中,$u_0(x,y),g(x,y,t)$是给定的函数且满足相容性条件,$\delta(x)$是Dirac函数,$(x_0,y_0)$选取为$\Omega$内一点,
$\rho(t)$举例可取为$\rho(t)=\sin(t)$。区域$\Omega$如图\ref{Domain}所示。\begin{figure}
    \centering
    \includegraphics[width=0.9\linewidth]{Domain.png}
    \caption{Domain}
    \label{Domain}
\end{figure}
\section{网格离散}
\section{数值格式}
\section{数值算例}
\section{数值结果展示分析}
\end{document}